% ~~~~~~~~~~~~~~~~~~~~~~~~~~~~~~~~~~~~~~~~~~~~~~~
% ~~~~~~~~~~~~~~ START OF PREAMBLE ~~~~~~~~~~~~~~

\documentclass[12pt]{article}
\usepackage{parskip} % for spacing between paragraphs and eliminating indentation at start of paragraph

% commands that will be "variables" for the rest of the document
\newcommand\groupname{\textit{Cometical}}
\newcommand\groupnumber{\textit{24}}
\newcommand\projectname{\textit{Assignment and Study Tracker\ }}
\newcommand\classname{\textit{CS 1200.005}}
\newcommand\assignment{\textit{Deliverable 2}}

% Place header for first page
\usepackage{fancyhdr}
\fancypagestyle{plain}{} % plain for first page
\fancyhf{}
\fancyhead[L]{\assignment}
\fancyhead[R]{\classname}

% Title 
\title{\groupname
\\ \small{group} \# \groupnumber}

% Group members, this is part of the title page
\author{
    \textbf{Diego R.R.} \\
    \texttt{dar220007@utdallas.edu} \\
    \and 
    \textbf{Gael Romero} \\
    \texttt{jgr230000@utdallas.edu} \\
    \and
    \textbf{Emily Rouse} \\
    \texttt{zxr220003@utdallas.edu} \\
    \and
    \textbf{Alan Roybal: \underline{leader}} \\
    \texttt{aer220004@utdallas.edu} \\
    \and
    \textbf{Rishabh Sabnavis} \\
    \texttt{rds230002@utdallas.edu}
}

\date{\today} % automatically generate date, part of title

% Setting header for subsequent pages
\pagestyle{fancy}
\fancyhead[L]{\assignment}
\fancyhead[R]{\classname}

% ~~~~~~~~~~~ END OF PREAMBLE ~~~~~~~~~~~
% ~~~~~~~~~~~~~~~~~~~~~~~~~~~~~~~~~~~~~~~


% ~~~~~~~~~~~~~~~~~~~~~~~~~~~~~~~~~~~~~~~
% ~~~~~~~~~~~~ START OF DOCUMENT ~~~~~~~~
\begin{document}
\maketitle % insert titlepage

% start of project description section
\section{Project Description} 
    We have selected the \projectname idea for our project. This app is envisioned as an integrative solution, tailored for students who are juggling multiple responsibilities and deadlines.

\subsection{Purpose and Integration with eLearning}
    The primary aim of this app is to provide a holistic approach to assignment and study management, working alongisde eLearning to introduce a customizable \projectname. By communicating with eLearning systems, our app seeks to work with various data points like assignment due dates, quiz schedules, test timings, and even potential event dates. This data, combined with our app's features, will lead to the generation of a customizable scheduling planner. The goal is to offer a platform that doesn't just remind students of their obligations but actively aids in time optimization based on these commitments.

\subsection{Customization} 
    One of the defining features of our app will be its ability to personalize the experience. We recognize that every student has their unique study habits, patterns, and preferences. While the student will provide details, such as when due dates of quizzes, assignments, tests, and even events to participate the app will suggest an intelligent scheduler that might look reasonable to the student to actually accomplish. This scheduler, using hollistic algorithms and past user interactions, will try to create a study plan tailored to the student's needs, helping them manage their time efficiently.

\subsection{Use cases}
    Imagine Joel, a Computer Science student, navigating his Software Engineering group project. With the app's help, he effortlessly inputs his project's divided tasks and gets prompted to allocate slots for collaborative group meetings. The tracker even optimizes these based on his past study patterns, ensuring he remains at his most productive. 

    In parallel, Sofia, a master's student in Computer Science, relies on the app to make sense of her regular assignments with an intensive research project. The tracker keeps her updated on research milestones and assists in dedicating post-research revision slots, ensuring she remains on top of both her academic and research commitments.
% end of project description

% start of requirements section
\section{Requirements for \projectname}
    \begin{enumerate}
        \item \textbf{Data Integration with eLearning Systems}: Ability to sync and retrieve information like assignment due dates, quiz schedules, test timings, and potential event dates.
        \item \textbf{Manual Data Entry}: Provide the option for students to manually input tasks or events not present in eLearning, such as group projects or external workshops.
        \item \textbf{Personalized Study Scheduling}: Generate time slot recommendations based on historical data and known student schedules, and adjust plans based on any last-minute changes or missed slots.
        \item \textbf{Study Habit Analysis}: Analyze and provide insights on the student's past study patterns, such as peak productivity periods.
        \item \textbf{Proactive Reminders}: Send notifications for upcoming tasks, commitments, and eLearning assignments or tests, with the option to set individual reminder timings.
        \item \textbf{Customizable Alert Preferences}: Allow students to customize their alert timings, and provide options to snooze or dismiss reminders.
        \item \textbf{Group Task Management}: Enable the creation, assignment, and sharing of tasks within study groups.
        \item \textbf{Event/Study Group Creation}: Provide tools to organize study sessions or review groups, and send invites to peers.
        \item \textbf{Personal Profile Settings}: Allow students to input their personal study preferences and adjust their schedule flexibility.
        \item \textbf{Theme and Appearance Customization}: Offer various app themes or layouts, and a choice between dark/light mode based on user preference.
        \item \textbf{Performance Analysis}: Enable visualization of study hours over time, track achievements, and review uncompleted or missed tasks.
    \end{enumerate}
% end of requirements section

% start of potential problems section
\section{Potential Problems}
    A standout concern with the Assignment and Study Tracker app is the user privacy and ensuring data security. The app's fundamental operation involves interpreting student data with study preferences ranging from academic to personal patterns. This comprehensive collection of data, while necessary for the app's effective functionality, runs the risk of becoming a vulnerability of sensitive information. Students are particularly sensitive about their academic data, and the fact that the app will have access to their academic performance, schedules, and possibly group interactions, magnifies the security potential problems. Moreover, as the app aims for deep integration with platforms like eLearning, the user data will potentially travel through unsafe networks.


     Morally wise, any breach or misuse could irreparably damage the trust of the user base and raise ethical concerns. The app's potential ability to capture specific patterns or insights about a student's habits could inadvertently lead to profiling or categorizing with negative connotations.

     To address this, the app should offer clear, concise, and easily accessible privacy policies, ensuring students are informed about the kind of data being collected, its use, storage policies, and the measures in place to protect it. Additionally, offering users control over what they share, and easy opt-out options, can help them and assuage concerns about potential overreach or violations of their personal boundaries.
% end of potential problems section


% start of subsystems section
\section{Subsystems}

    \subsection{Data Integration and Synchronization}
    This subsystem handles the connection between the app and the eLearning platform. It ensures real-time syncing of assignment due dates, quiz schedules, test timings, and other academic events. This subsystem needs to be robust to prevent any data mismatches or lags, which could misinform students about their schedules.

    \subsection{Scheduling and Planning}
    Central to the app's value proposition, this subsystem employs machine intelligence techniques and historical user interaction data to generate personalized study and assignment schedules. It optimizes study time, integrates breaks, and aims to adapt to the user's most productive timeframes and personal preferences.

    \subsection{Notification, Alerts and Data Interpretation}
    Basically the interpreter that handles the user's input. It allows users to input their study preferences such as their most productive hours, and parse their response to generate data primites such as classes and structures used throughout the program.

    \subsection{User Profile Management and Customization}
    User experience will define storing and managing user-specific data, including study habits, preferred study times, break durations, and past academic performance. It will also be the most interactive subsystem with the user.

    \subsection{Security and Privacy}
    Given the sensitivity of the data involved, this subsystem oversees data encryption, secure data storage, and safe data transmission. It also manages user consent, ensuring that users are aware of the data the app collects and have granular control over this process.
% end of subsystems section

\end{document}
